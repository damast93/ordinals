\documentclass[12pt,a4paper,]{article}

% Random Verbesserungen und Sprache
\usepackage{fixltx2e}

\usepackage[ngerman]{babel}
\usepackage[babel, german=guillemets]{csquotes}

% Typografie
\usepackage[T1]{fontenc}
\usepackage{xcolor}
\usepackage{titlesec}

\usepackage{pdfpages}

\usepackage[condensed,math]{anttor}
\usepackage{libertine}
\setmainfont[Ligatures=TeX]{Linux Libertine O}

\usepackage{fontspec}

\usepackage{etoolbox}
\newfontfamily\quotefont{Antykwa Torunska Condensed}
\AtBeginEnvironment{quote}{\quotefont}

\usepackage{microtype}

% Symbole
\usepackage{amsmath}
\usepackage{amsfonts}
\usepackage{amssymb}

% Kommutative Diagramme
\usepackage[all,cmtip]{xy}

% Bibliografie
\usepackage{doc}
\usepackage[style=alphabetic]{biblatex} 
\renewcommand*{\intitlepunct}{}
\DefineBibliographyStrings{german}{in={}}

% Layout-Setup
\usepackage[left=3cm,right=3cm,top=1.5cm,bottom=1.2cm,includeheadfoot]{geometry}

% Grafiken erstellen und Einbinden
\usepackage{tikz}
\usepackage{graphicx}

% Beweis-Umgebungen
\usepackage[standard,amsmath,thmmarks]{ntheorem}

% Fortlaufende per-Section-Nummerierung (mit dummy-Counter)
\theoremstyle{margin}
\theorembodyfont{\normalfont}
\newtheorem{Schritt}{Schritt}
\newtheorem{counter}{counter}[section]

\renewtheorem{Beispiel}[counter]{Beispiel}
\renewtheorem{Definition}[counter]{Definition}
\renewtheorem{Bemerkung}[counter]{Bemerkung}
\renewtheorem{Proposition}[counter]{Frage}

\theorembodyfont{\slshape}
\renewtheorem{Korollar}[counter]{Korollar}
\renewtheorem{Lemma}[counter]{Lemma}
\renewtheorem{Satz}[counter]{Satz}

\theoremstyle{nonumberplain}
\theoremsymbol{\ensuremath{\,\hfill\square}}
\theoremheaderfont{\normalfont\scshape}
\theorembodyfont{\normalfont}
\theoremseparator{:} 
\renewtheorem{Beweis}{Beweis}

% Mathe-Abkürzungen, Symboldefinitionen
\newcommand{\id}{\textrm{id}}
\newcommand{\Id}{\textrm{Id}}
\renewcommand{\hom}{\textrm{Hom}}
\newcommand{\op}{\textrm{op}}
\newcommand{\im}{\textrm{im}}
\newcommand{\tensor}{\otimes}
\newcommand{\coker}{\textrm{coker}}
\newcommand{\cat}{\textbf}
\newcommand{\iso}{\cong}
\newcommand{\hask}[1]{\texttt{#1}}
\newcommand{\nitem}[1]{\item\textbf{#1}}
\newcommand{\diag}[1]{$$\xymatrix{#1}$$}
\renewcommand{\bf}{\mathbf}


% Hyperlinks
\providecommand\phantomsection{} % Braucht man, damit hyperref mit section* klarkommt
\usepackage[colorlinks=true, linkcolor=blue, citecolor=blue]{hyperref}
\hypersetup{
    hypertexnames=false,
    pdfauthor={Dario Stein},
}



\title{Reminder on ordinal arithmetic}

\newcommand{\cf}{\mathrm{cf}}

\begin{document}

\maketitle

\part{Definitions}
\section{Well-orders}
Ordinals are the mother of all well-orders. A total order is called a \emph{well-order} if one the following equivalent conditions applies
\begin{itemize}
	\item Any nonempty set of element has a least element
	\item There is no infinite descending sequence of elements
\end{itemize}
Every subset of a well-ordered set is again well-ordered. Every element $a \in P$ that is not maximal has an immediate successor $a^+ = \min \{ b : b > a \}$. Every bounded subset of a well-order has a supremum. For every element $a \in P$, $a$ is either the minimal element, a \emph{successor element} ($a = b^+$) or a so-called \emph{limit element}. In this case, it is the supremum of the elements strictly below it. By the axiom of choice, every set can be well-ordered. \\

\textbf{Proposition} [Trichotomy]
If $P,Q$ are two well-orders, then we have the following trichotomy
\begin{itemize}
	\item $P$ is order-isomorphic to a proper initial segment of $Q$
	\item $P$ is order-isomorphic to $Q$
	\item A proper initial segment of $P$ is order-isomorphic to $Q$.
\end{itemize}

We call two well-orders \emph{of the same order-type} if they are order-isomorphic. The order-types are thus totally ordered by `is isomorphic to an initial segment of'. This will turn out to be a well-order.

\section{Ordinals}
Ordinal numbers are certain well-ordered sets, serving as distinguished representatives for order-types. For every well-ordered set $P$, there is a unique ordinal $\alpha$ such that $\alpha$ and $P$ are order-isomorphic. The proper class of ordinal numbers is well-ordered, and every well-ordered set is order-isomorphic to an initial segment of the ordinal numbers.

We use the von Neumann-construction of ordinals, i.e. for every ordinal $\lambda$, we let

\[ \lambda = \{ \alpha : \alpha < \lambda \} = [0,\lambda) \]

as a set. We have for all ordinals $\alpha,\lambda$ that
\[ \alpha < \lambda \Leftrightarrow \alpha \in \lambda \Leftrightarrow \alpha \subset \lambda \Leftrightarrow \alpha \text{ proper initial segment of } \lambda \]
The ordinals are as sets defined recursively via

\[ 
\begin{cases}
	0 = \emptyset \\
	\alpha^+ = \alpha \cup \{\alpha\} \\
	\lambda = \bigcup_{\alpha < \lambda} \alpha
\end{cases}
\]

For example we have $2 = \{0,1\} = \{0,\{0\}\} = \{\emptyset, \{\emptyset\}\}$ and we get the first transfinite ordinal $\omega$ via
\[ \omega := \bigcup_{n \text{ finite }} n = \{0,1,2,\ldots\} = \mathbb N \]
and its successor
\[ \omega^+ := \mathbb N \cup \{ \mathbb N \} = \{0,1,2,\ldots\, \mathbb N \}. \]

Every \emph{set} $\Gamma$ of ordinals has a supremum in the ordinals, and we use the following three notations interchangably.

\[ \sup \Gamma = \lim_{\gamma \in \Gamma} \gamma := \bigcup_{\gamma \in \Gamma} \gamma \]

\section{Cardinals}
We model cardinals as special ordinals, namely we identify every cardinal $\mathfrak a$ with the least ordinal $\alpha$ that is in bijection with $\mathfrak a$. Note that now the cardinal $\alpha$ is itself the canonical representative of a set of cardinality $\alpha$, as
\[ |\alpha| = |\{ \beta : \beta < \alpha \}| = \alpha. \]

Note that $\omega$ is the first infinite ordinal and thus a cardinal
\[ \omega =: \aleph_0 \]

Cardinals are again well-ordered, leading to their own notion of cardinal sucessors. We can thus enumerate the infinite cardinals by ordinals
\[ 
\begin{cases}
	\aleph_0 = \omega \\
	\aleph_{\alpha^+} = (\aleph_\alpha)^+ \\
	\aleph_\lambda = \bigcup_{\alpha < \lambda} \aleph_\alpha
\end{cases}
\]

We let $\omega_{\alpha} = \aleph_{\alpha}$, so e.g. $\omega_1$ is the first uncountable ordinal. \\

\textbf{Nonexample:} Let's appreciate some of the non-intuitiveness of $\omega_1$. It is an uncountable well-ordered set, but every initial segment is countable. Unlike $\omega = \sup \{0,1,2,\ldots\}$, $\omega_1$ is not the supremum of any countable subset. Let $\omega_1 = \sup \Gamma, \Gamma \subseteq \omega_1$ countable, then
\[ \omega_1 = \bigcup \Gamma, \] thus $\omega_1$ is countable union of countable sets, hence countable; contradiction. 

\section{Cofinality}
The phenomenon from above is captured by the notion of cofinality. A subset $N \subseteq P$ of an ordered set is called \emph{cofinal} if 
\[ \forall p \in P \exists n \in N : p \leq n. \]
The cofinality $\cf(P)$ is the least cardinality of a cofinal subset of $P$. We have $\cf(P) = 1$ iff $P$ has a maximial element, otherwise at least $\cf(P) \geq \omega$. \\

For an ordinal $\alpha$, $\cf(\alpha)$ is the least cardinality of a subset $\Gamma \subseteq \alpha$ with $\alpha = \sup \Gamma$. We have shown that $\cf(\omega_1) = \omega_1$. \\

A cardinal $\lambda$ is called \emph{regular} iff $\cf(\lambda) = \lambda$. \\

\textbf{Proposition: } For every successor cardinal $\alpha$, $\aleph_\alpha$ is regular. For every limit ordinal $\lambda$, we have $\cf(\aleph_\lambda) = \cf(\lambda)$ instead. \\

\textbf{Example: } $\aleph_\omega$ is not regular, as 
\[ \cf(\aleph_\omega) = \cf(\omega) = \omega. \] 
In fact, by definition $\aleph_\omega = \sup \{ \aleph_0, \aleph_1, \ldots \}$ is supremum of a countable subset. 

\section{Topology}
On every totally ordered set, there is an induced topology called the \emph{order topology} given by the basis of open rays and open intervals. For example, the order topology on $\mathbb R$ with the usual order is just the euclidean one.

Every ordinal $\alpha$ becomes a topological space of its own in the order topology. For example the ordinal $\omega$ is homeomorphic to the subspace $\mathbb N \subset \mathbb R$, whereas $\omega^+$ is homeomorphic to
\[ \{ 1 - 1/n : n = 1,2, \ldots\} \cup \{ 1 \}. \]
At some point, larger ordinals will not be metric spaces any more. \\


\pagebreak 

\part{Ordinal arithmetic}
\section{Addition}
Ordinal arithmetic is defined recursively with recursion on the right operand. This leads to a certain asymmetry and enforces certain rules only on the right side.

\[ 
\begin{cases}
	\alpha + 0 = \alpha \\
	\alpha + (\beta^+) = (\alpha + \beta)^+ \\
	\alpha + \lambda = \bigcup_{\beta < \lambda} \alpha + \beta
\end{cases}
\]

\textbf{Proposition: } Addition is strictly monotonic and continuous in the right argument. Addition is associative

\textbf{Non-examples: } Addition is merely non-decreasing in the left argument and not continuous. Addition is not commutative.

We have 
\[ 2 + \omega = \bigcup_{n < \omega} 2 + n = \sup \{ 2, 3, 4, \ldots \} = \omega. \]
From there, we get $0 + \omega = \omega = 2 + \omega$ and $\omega + 2 \neq 2 + \omega$. Regarding continuity
\[ \omega + 2 = \sup \{0,1,2,\ldots\} + 2 \neq \sup \{0 + 2, 1 + 2, \ldots\} = \omega. \]

\textbf{Proposition} Let $P,Q$ be two well-orders with order types $\alpha,\beta$. Then $\alpha+\beta$ is the order type of the well-order on $P \sqcup Q$ where every element of $P$ comes before every element of $Q$. E.g.

\[ \omega + \omega + 1 \cong \{ 0 < 1 < 2 < \ldots 0' < 1' < 2' < \ldots 0'' \} \]

\section{Ordinal multiplication}
We define multiplication by continuity and distributivity on the right.

\[ 
\begin{cases}
	\alpha \cdot 0 = 0 \\
	\alpha \cdot (\beta + 1) = (\alpha \cdot \beta) + \alpha \\
	\alpha \cdot \lambda = \bigcup_{\beta < \lambda} \alpha \cdot \beta
\end{cases}
\]

\textbf{Proposition: } Multiplication is strictly monotonic and continuous on the right and associative. $1$ is both-sided identity.

\textbf{Non-examples: } Multiplication is merely non-decreasing in the left argument and not continuous. Multiplication is not commutative and not distributive in the left argument. \\

We have $\omega \cdot 2 = \omega + \omega$, whereas

\[ 2 \cdot \omega = \bigcup_{n<\omega} 2\cdot n = \sup \{ 0, 2, 4, \ldots \} = \omega. \]

Therefore we get $\omega \cdot 2 \neq 2 \cdot \omega$ and $1\cdot \omega = 2\cdot \omega \neq (1+1)\times \omega$. Regarding continuity, 
\[ \omega \cdot \omega = \sup \{ 1, 2, 3, \ldots \} \cdot \omega \neq \sup \{ 1 \cdot \omega, 2 \cdot \omega, 3 \cdot \omega, \ldots \} = \sup \{ \omega, \omega, \omega, \ldots \}. \]

\textbf{Proposition: } Let $P,Q$ be two well-orders with order types $\alpha,\beta$. Then $\alpha \cdot \beta$ is the order type of $P \times Q$ regarded as $Q$ copies of $P$ (reverse lexical order), i.e.

\[ \omega\cdot 2 \cong \{ 0 < 1 < 2 < \ldots 0' < 1' < 2' < \ldots \} \cong \omega + \omega \]

whereas

\[ 2 \cdot \omega \cong \{ 0 < 1 < 0' < 1' < 0'' < 1'' < \ldots \} \cong \omega \]

\section{Ordinal exponentiation}
Ordinal exponentation is defined recursively with continuity and the power-law $\alpha^{\beta + \gamma} = \alpha^\beta \cdot \alpha^\gamma$ in mind.

\[ 
\begin{cases}
	\alpha^0 = 1 \\
	\alpha^{\beta + 1} = \alpha^\beta \cdot \alpha \\
	\alpha^\lambda = \bigcup_{\beta < \lambda} \alpha^\beta
\end{cases}
\]

\textbf{Proposition} Exponentation is strictly monotonic and continuous in the exponent. It respects multiplication in the exponent, i.e. $\alpha^{\beta\gamma} = (\alpha^\beta)^\gamma$.

\textbf{Non-examples: } Exponentiation is merely non-decreasing and not continuous in the base. It does not respect multiplication in the base.

First of all, for all $1 < k < \omega$ we have
\[ k^\omega = \sup \{ k^n : n < \omega \} = \omega \]
Thus $2^\omega = 3^\omega$. Also 
\[ \omega^2 \neq \sup \{ n^2 : n < \omega \} = \omega. \]
Regarding multiplication in the base, using associativity, we get
\[ (\omega \cdot 2)^2 = (\omega \cdot 2)(\omega \cdot 2) = \omega \cdot (2\omega) \cdot 2 = \omega \cdot \omega \cdot 2 = \omega^2 \cdot 2. \]

But then $(\omega \cdot 2)^2 \neq \omega^2 \cdot 2^2 = \omega^2 \cdot 4$. 

\textbf{Proposition}
Let $P,Q$ be two well-orders with order types $\alpha, \beta$. Then $\alpha^\beta$ is the order type of the set of function with finite support $\alpha^{(\beta)}$ in reverse lexical order.

\section{Fixed-points}
Consider the power towers
\[ \omega \uparrow\uparrow 1 = \omega, \omega \uparrow \uparrow (n + 1) = \omega^{\omega \uparrow \uparrow n}. \]
We define their supremum as 
\[ \epsilon_0 := \sup \{ \omega \uparrow \uparrow n : n < \omega \}. \]
\textbf{Proposition: } $\epsilon_0$ satisfies the equation $\omega^{\epsilon_0} = \epsilon_0$.

By continuity in the exponent, we get
\[ \omega^{\epsilon_0} = \sup_{n < \omega} \left(\omega^{\omega \uparrow \uparrow n}\right) = \sup_{n < \omega} (\omega \uparrow \uparrow (n+1)) = \epsilon_0 \]
and $\epsilon_0$ is the least fixed point of $\omega^\cdot$. \\ 

\textbf{Proposition} $\epsilon < \omega_1$, i.e. $\epsilon_0$ is still countable.
This is because for infinite ordinals of cardinality at most $\kappa$, all arithmetic operations still produce ordinals of size at most $\kappa$. Note that by our constructions, the resulting ordinals are in bijection with just sums, products and functions with finite support of their underlying sets. 
Now
\[ \epsilon_0 = \bigcup_{n < \omega} \omega \uparrow \uparrow n \]
is countable union of countable sets, thus countable. \\

Note that we can produce other fixed-points for the continuous arithmetical operations in the very same fasion

\begin{align*}
\omega^{\epsilon_0} &= \epsilon_0 \\
\omega\cdot \omega^\omega &= \omega^\omega \\
\omega + \omega^2 &= \omega^2 \\
1 + \omega &= \omega.
\end{align*}

\section{Cantor normal form}
The ordinals in the interval $[0,\epsilon_0)$ are closed under ordinal addition, multiplication and exponentation. Every number $\alpha$ in that interval can be expressed as a unique so-called \emph{Cantor normal form} (CNF). 
\[ \alpha = \omega^{\beta_1}c_1 + \ldots + \omega^{\beta_n}c_n \]
where $\alpha > \beta_1 > \ldots > \beta_n$ and $c_i \in \mathbb N$. 
We can recursively transform the exponents $\beta_i$ in CNF and get a hereditary representation in base $\omega$ after finitely many steps. We see that $[0,\epsilon_0)$ is actually generated by the natural numbers and $\omega$ under the arithmetical operations.  Arithmetic in CNF has simple computational descriptions.

\subsection{Order}
CNFs can be compared lexicographically. Highest exponents first, then coefficients. 

\subsection{Addition}
We see that
\[
\omega^\beta c + \omega^{\beta'}c' = \begin{cases}
	\omega^\beta c + \omega^{\beta'}c' & \text{ if } \beta > \beta' \\
	\omega^\beta (c+c') & \text{ if } \beta = \beta' \\
	\omega^{\beta'}c' & \text{ if } \beta < \beta'
\end{cases}
\]
If exponents are decreasing, they are already in CNF. If exponents are the same, simplify. If exponents are increasing, the bigger term absorbs the smaller one.
\textbf{Example}
\[ \omega^2 + \omega^3 = \omega^2 + \omega^2\omega = \omega^2(1+\omega) = \omega^2\omega = \omega^3. \]
Thus in order to add a term to a CNF, insert it at the correct position, simplify and drop all further terms.

\subsection{Multiplication}
Multiplication is distributive on the right. Thus we just need to understand how to multiply a CNF with a single term on the right. Let
\[ \alpha = \omega^{\beta_1}c_1 + \ldots + \omega^{\beta_n}c_n \]
then for $\beta > 0$
\[ \alpha \cdot \omega^\beta = (\omega^{\beta_1}c_1 + \ldots + \omega^{\beta_n}c_n) \cdot \omega^\beta = \omega^{\beta_1 + \beta}. \]
Through repeated summation, we get for $n \in \mathbb N \setminus \{0\}$
\[ \alpha \cdot n = \omega^{\beta_1}c_1 n + \ldots + \omega^{\beta_n}c_n. \]
So multiplication just acts on the highest term. \textbf{For example}
\[ (\omega^3 + \omega^2)\cdot 3 = \omega^3 + \omega^2 + \omega^3 + \omega^2 + \omega^3 + \omega^2 = \omega^3\cdot 3 + \omega^2. \]
In the limit case, we get
\begin{align*}
 (\omega^3 \cdot 2 + \omega^2)\omega &= \sup_{n < \omega} (\omega^3 \cdot 2 + \omega^2)n \\
 &= \sup_{n < \omega} (\omega^3 \cdot 2n + \omega^2) \\
 &= \omega^3 \omega = \omega^4.
\end{align*}

\subsection{Exponentiation}
Exponentiation with base $\omega$ is easy. For any $\beta$ in CNF, $\omega^\beta$ is already in CNF. Exponentiation with arbitrary base is more complicated. Let

\[ \alpha = \omega^{\beta_1}c_1 + \ldots + \omega^{\beta_n}c_n \]

By the power laws
\[
 \alpha^{\beta + \gamma} = \alpha^\beta \cdot \alpha^\gamma, \quad \alpha^{\beta \gamma} = (\alpha^\beta)^\gamma, \]
we only need to unterstand taking powers with a single term. We distinguish finite and transfinite terms. 

\subsubsection{Finite exponents}
For $0 < r < \omega$, we can compute $\alpha^r$ by repeated multiplication. \\

\subsubsection{Transfinite exponents}
For exponents $\omega^\beta$ with $\beta > 0$, first distinguish if $\alpha$ is finite. If it is, note that
\[ \alpha^\omega = \omega, \]
so if $\beta$ is finite, then
\[ \alpha^{\omega^\beta} = \omega^{\omega^{\beta-1}}. \]
If $\beta$ is transfinite, then $\beta = 1 + \beta$, so
\[ \alpha^{\omega^\beta} = \alpha^{\omega^{1+\beta}} = \alpha^{\omega\omega^\beta} = \omega^{\omega^\beta}. \]
If $\alpha$ is transfinite, we simply get 
\[ \alpha^{\omega^\beta} = \omega^{\beta_1\omega^\beta}. \]

\textbf{Example}
\begin{align*}
(\omega^2 \cdot 2)^\omega &= \sup_{n<\omega} (\omega^2 \cdot 2)^n \\
&= \sup_{n < \omega} \omega^{2n}\cdot 2 \\
&= \omega^\omega. 
\end{align*}

\section{Goodstein's sequence}
Goodstein's theorem is a very surprising theorem in number theory. Fix a basis $b \geq 2$. Take any number $n$ and write it in base-$b$,
\[ n = b^{k_1}c_1 + \ldots + b^{k_\ell} c_\ell. \]
Now $n > k_1 > \ldots > k_\ell$, so we can write the exponents in base-$b$ again and obtain a hereditary base-$b$ representation. \textbf{Example}
\[ 13 = 8 + 4 + 1 = 2^3 + 2^2 + 1 = 2^{2+1} + 2^2 + 1. \]

Fix a number $n$. We define the \emph{Goodstein sequence} $g^n(2), g^n(3), \ldots$ recursively. Let $g^n(2) = n$. Now write $g^n(2)$ in hereditary base-$2$ and bump the base, i.e. replace all $2$s by $3$s. Subtract one. That gives us $g^n(3)$. Repeat that, so in general
\[ g^n(b+1) = \text{ write $g^n(b)$ in base-$b$, replace $b\to b+1$, subtract $1$}. \]
 \textbf{For example}
\begin{align*}
	g^4(2) &= 4 = 2^2 \\
	g^4(3) &= 3^3 - 1 = 26 \\
	       &= 3^2\cdot 2 + 3\cdot 2 + 2 \\
	g^4(4) &= 4^2 \cdot 2 + 4 \cdot 2 + 1 \\
	g^4(5) &= 5^2 \cdot 2 + 5 \cdot 2 \\
	g^4(6) &= 6^2 \cdot 2 + 6 \cdot 2 - 1 \\
	       &= 6^2 \cdot 2 + 6 + 5 \\
	g^4(7) &= 7^2 \cdot 2 + 7 + 4
\end{align*}
The numerical values of this sequence are
\[ 4, 26, 41, 60, 83, 109, \ldots \]
\textbf{Theorem: }[Goodstein] For every integer $n$, the Goodstein sequence $g^n$ eventually becomes $0$. \\

The thing is: This takes long, extreemly, ridiculously long. The Goodstein function $G : \mathbb N \to \mathbb N$
\[ G(n) = \min \{ b : g^n(b) = 0 \} \]
is one of the fastest-growing functions ocurring in mathematics. We have
\[ G(3) = 7, \] that is $g^3$ terminates at base $7$. How long might $g^4$ need? We have
\[ G(4) = 3\cdot 2^{402653211}-1. \]
That is a number with roughly 121 million \emph{decimal digits}. The numbers of the sequence are of the same order of magnitude. Even though the numbers seem to get larger and larger, something forces them to become 0 at the end. They seem to lose some complexity throughout the operations. We can precisely pin down this complexity using ordinals and CNF. \\

\textbf{Proof of Goodstein's theorem} We know how the number $g^n(b)$ can be written in hereditary base-$b$. Define a second sequence $w^n(b)$ of ordinals in CNF by taking that representation and replacing $b$ with $\omega$. For example

\begin{align*}
	w^4(2) &= \omega^\omega \\
	w^4(3) &= \omega^2 \cdot 2 + \omega \cdot 2 + 2 \\
	w^4(4) &= \omega^2 \cdot 2 + \omega \cdot 2 + 1 \\
	w^4(5) &= \omega^2 \cdot 2 + \omega \cdot 2 \\
	w^4(6) &= \omega^2 \cdot 2 + \omega + 5
\end{align*}

This sequence of ordinals is strictly decreasing, thus can only have finite length. So $g^n$ can only have finitely many elements before stabilizing at zero.
\end{document}