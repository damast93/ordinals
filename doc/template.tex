\documentclass[12pt,a4paper,]{article}

% Random Verbesserungen und Sprache
\usepackage{fixltx2e}

\usepackage[ngerman]{babel}
\usepackage[babel, german=guillemets]{csquotes}

% Typografie
\usepackage[T1]{fontenc}
\usepackage{xcolor}
\usepackage{titlesec}

\usepackage{pdfpages}

\usepackage[condensed,math]{anttor}
\usepackage{libertine}
\setmainfont[Ligatures=TeX]{Linux Libertine O}

\usepackage{fontspec}

\usepackage{etoolbox}
\newfontfamily\quotefont{Antykwa Torunska Condensed}
\AtBeginEnvironment{quote}{\quotefont}

\usepackage{microtype}

% Symbole
\usepackage{amsmath}
\usepackage{amsfonts}
\usepackage{amssymb}

% Kommutative Diagramme
\usepackage[all,cmtip]{xy}

% Bibliografie
\usepackage{doc}
\usepackage[style=alphabetic]{biblatex} 
\renewcommand*{\intitlepunct}{}
\DefineBibliographyStrings{german}{in={}}

% Layout-Setup
\usepackage[left=3cm,right=3cm,top=1.5cm,bottom=1.2cm,includeheadfoot]{geometry}

% Grafiken erstellen und Einbinden
\usepackage{tikz}
\usepackage{graphicx}

% Beweis-Umgebungen
\usepackage[standard,amsmath,thmmarks]{ntheorem}

% Fortlaufende per-Section-Nummerierung (mit dummy-Counter)
\theoremstyle{margin}
\theorembodyfont{\normalfont}
\newtheorem{Schritt}{Schritt}
\newtheorem{counter}{counter}[section]

\renewtheorem{Beispiel}[counter]{Beispiel}
\renewtheorem{Definition}[counter]{Definition}
\renewtheorem{Bemerkung}[counter]{Bemerkung}
\renewtheorem{Proposition}[counter]{Frage}

\theorembodyfont{\slshape}
\renewtheorem{Korollar}[counter]{Korollar}
\renewtheorem{Lemma}[counter]{Lemma}
\renewtheorem{Satz}[counter]{Satz}

\theoremstyle{nonumberplain}
\theoremsymbol{\ensuremath{\,\hfill\square}}
\theoremheaderfont{\normalfont\scshape}
\theorembodyfont{\normalfont}
\theoremseparator{:} 
\renewtheorem{Beweis}{Beweis}

% Mathe-Abkürzungen, Symboldefinitionen
\newcommand{\id}{\textrm{id}}
\newcommand{\Id}{\textrm{Id}}
\renewcommand{\hom}{\textrm{Hom}}
\newcommand{\op}{\textrm{op}}
\newcommand{\im}{\textrm{im}}
\newcommand{\tensor}{\otimes}
\newcommand{\coker}{\textrm{coker}}
\newcommand{\cat}{\textbf}
\newcommand{\iso}{\cong}
\newcommand{\hask}[1]{\texttt{#1}}
\newcommand{\nitem}[1]{\item\textbf{#1}}
\newcommand{\diag}[1]{$$\xymatrix{#1}$$}
\renewcommand{\bf}{\mathbf}


% Hyperlinks
\providecommand\phantomsection{} % Braucht man, damit hyperref mit section* klarkommt
\usepackage[colorlinks=true, linkcolor=blue, citecolor=blue]{hyperref}
\hypersetup{
    hypertexnames=false,
    pdfauthor={Dario Stein},
}

